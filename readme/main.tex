\documentclass[UTF8,fontset=windowsnew]{ctexart}
\usepackage{amsmath}
\DeclareMathOperator*{\uint}{\scalerel*{\rotatebox{8}{$\!\scriptstyle\int\!$}}{\int}}
  \usepackage{scalerel}
\usepackage{hyperref}
\usepackage{graphicx}
\usepackage{xcolor}
\usepackage[left=2cm,right=2cm, top=2cm, bottom=2cm]{geometry}
\usepackage{fancyhdr}
  \pagestyle{fancy}
  \fancyhf{}
  \rhead{\large{\emph{考核系统}}}
  \setlength{\headheight}{20pt}
  \rfoot{\thepage}
\usepackage{enumitem}
\usepackage{tikz}
  \newcommand*\circled[1]{\tikz[baseline=(char.base)]{\node[shape=circle,draw,inner sep=2pt] (char) {#1};}}
  \newcommand{\RomanNumeralCaps}[1]{\MakeUppercase{\romannumeral #1}}
\usepackage{multicol}

\begin{document}
\songti

\section{需求}
\subsection{原始需求}
需要对每个被考核人进行工作量计算。
\begin{enumerate}[label=\protect\circled{\arabic*}]
    \item 工作量计算
    \item 计算过程图像化展示
    \item 用户管理
\end{enumerate}
\subsection{需求分析}
\subsubsection{工作量计算}
\begin{equation}
  C=\frac{1}{N}\sum_{i-1}^NC_i\label{eq:main}
\end{equation}
$C_i=\frac{W_i}{S_i}$为总工作量,$N$为考核项目数量,$W_i$为考核项目完成量,$S_i$为标准任务量。\par
\subsubsection{图像化展示}
将各项参数的组成部分图像化
\subsubsection{用户管理}
允许被考核人亲自上传各个表格
\section{被考核人}
被考核人应由工号辨别,如允许以姓名辨别,应处理重名。\par
每个被考核人属于一种岗位类型:\par
\begin{itemize}
  \item 教学科研岗
  \item 教学岗
  \item 科研岗
\end{itemize}
每个岗位类型分10级,不同等级对应不同的基准工作量和总任务量。
\section{工作量计算标准}
每种考核方向的比重相同,以恰好完成基准工作量为1,如必要项未完成,为0;如非必要项未完成,按比例扣分;如超标完成,按比例加分。\par
完成每种考核后按\autoref{eq:main}计算总分。\par
\section{产品介绍}
由于工期原因,仅需求\circled{1}可以保证完成;如一切顺利,需求\circled{2}可以数字的形式显示,图像化无法完成;需求\circled{3}确定无法按期完成。\par
\subsubsection{基本架构}
\begin{figure}[h]
  \centering
  \includegraphics[width=.5\textwidth]{image/struc.png}
  \caption{基本架构}
  \label{fig:struc}
\end{figure}
如\autoref{fig:struc}所示,本产品的基本模式为服务器-客户端模式。\par
\subsection{服务器}
服务器采用Node.js/Express.js/Pug技术,可在Windows/Linux平台上运行。\par
服务器的运行依赖于Nodejs和npm。\par
默认端口为10000。\par
如果需要,可以提供服务器的源代码。\par
测试服务器的地址为\url{http://34.94.165.54:10000/}\par
测试服务器预计可以提供服务到2020年9月。期间如果系统出问题可以保证24小时内解决。
\subsection{客户端}
本产品的客户端为任何浏览器,支持IE 8+, Chrome, Firefox等主流浏览器。如果网页显示出现问题建议更换浏览器后重试。\par
界面:\par
\begin{figure}[h]
  \centering
  \includegraphics[width=.5\textwidth]{image/client.png}
  \caption{主页面}
  \label{fig:client}
\end{figure}
如\autoref{fig:client}所示,主界面拥有4个链接。\par
\begin{itemize}[font=\emph,leftmargin=1cm]
  \item [选择表格] 点击该项会弹窗,请选择需要上传的表格
  \item [使用本系统前请阅读此文] 该项为本文的链接
  \item [下载模板表格] 点击该项会下载模板输入表格,请根据模板表格的结构修改输入数据
  \item [上传表格] 点击该按钮会上传输入表格并开始计算。由于输入文件较复杂,需要花费较长时间。请耐心等待10秒。
\end{itemize}
\subsection{输入}
\url{http://34.94.165.54:10000/template}为模板输入文件。其中有12个表格。每个表格有若干列数据列。数据列以外的列不会影响计算,可以随意编辑。\par
数据列第一行不允许改动,否则系统无法识别数据。其它列没有限制。\par
数据列每一列皆有特定的规范,具体请见模板中每一页的注释。\par
请尤其注意,各表中姓名的识别方式不同,如果姓名的格式错误可能导致该行数据无法识别。
\subsection{输出}
表格上传并计算完毕后会自动下载输出表格,下载完成后打开输出表格,第一页的G列为输出。\par
\begin{figure}[h]
  \centering
  \includegraphics[width=.5\textwidth]{image/output.png}
  \caption{输出范例}
  \label{fig:output}
\end{figure}
\section{缺点}
由于教学为主的老师只有一位,其分数组成又和其他类型的老师差别很大,因此其算法很可能错误。请手动计算其分数。\par
所有团体活动的分数均设为1。如果有详细数据请提供。\par
未纳入班主任和课程设计数据,请手动将其折算并填入本科教学表格。\par
\section{疑问}
以下是对输入数据和算法的疑问。
\begin{enumerate}
  \item 学生参加创新竞赛或学科竞赛的数据未找到。请指出其数据来源
  \item 在如``发表SCI期刊论文2篇,EI期刊论文或重要核心期刊论文1篇''的要求中,如果SCI没有达到要求,是否无论EI有多少篇都算未完成?
  \item 教学成果奖仅在研究生教改、成果表格中找到,请问本科是否没有教学成果奖?
  \item 若要求是发表SCI论文,是否只计算SCI和SCI-EI而不管EI?如果要求是EI,是否只统计EI而不管SCI和SCI-EI?
  \item SCI-EI是否等于2个SCI?
  \item 专利数据中姓名格式极为混乱,姓名间没有分隔符,姓和名间有的还有空格。请提供姓名由/分隔,且无空格的数据,如果必须使用其它分隔符或格式,请告知。
  \item 模板为我认为所有有效数据,如有必须考虑但模板中未纳入的数据,请提供其数据和算法介绍。
\end{enumerate}
\end{document}
